\documentclass[12pt, a4paper]{article}
\usepackage[utf8]{inputenc}
\usepackage[left=3.00cm,
            right=2.00cm,
            top=3.00cm,
            bottom=2.00cm]{geometry}
\usepackage[brazilian]{babel} % Hifenização e dicionário
\usepackage{enumitem}         % Para itemsep etc
\usepackage{zi4}              % Para fonte de códigos
\usepackage{listings}         % Para códigos
\usepackage{lstautogobble}    % Códigos indentados corretamente
\usepackage{color}            % Para coloração de códigos
\usepackage{mathpazo}         % Palatino
\usepackage{parskip}          % Linha em branco entre parágrafos em vez de recuo
\usepackage{verbatim}         % Para comentários
\usepackage{amsmath}
\usepackage{graphicx}
\usepackage{booktabs}
\usepackage[breaklinks]{hyperref}

\DeclareGraphicsExtensions{.pdf,.png}

\newcommand{\code}[1]{{\lstinline{#1}}}

\usepackage{listings}
\lstset{
    autogobble,
    columns=fullflexible,
    showspaces=false,
    showtabs=false,
    breaklines=true,
    showstringspaces=false,
    breakatwhitespace=true,
    escapeinside={(*@}{@*)},
    basicstyle=\ttfamily\footnotesize,
    frame=l,
    framesep=12pt,
    xleftmargin=12pt,
    tabsize=4,
    captionpos=b
}

\DeclareMathOperator*{\argmax}{arg\,max}
\DeclareMathOperator*{\argmin}{arg\,min}

\begin{document}
\begin{center}
    \textsc{Universidade Federal do Rio Grande do Norte} \\
    \textsc{Departamento de Informática e Matemática Aplicada}
\end{center}

\bigskip

\begin{tabular}{@{}rl@{}}
    \emph{Disciplina} & Processamento de Linguagem Natural \\
    \emph{Docente}    & Carlos Augusto Prolo \\
    \emph{Discente}   & Felipe Cortez de Sá \\
\end{tabular}

\bigskip

\begin{center}
\large Extração de gramática
\end{center}

\section{Introdução}
Este relatório descreve a implementação de um programa para extrair a gramática
de um corpus e convertê-la para a \emph{Chomsky Normal Form}.

\section{Programa}
O programa foi desenvolvido utilizando a linguagem de programação Python 3
apenas com as bibliotecas padrão, sendo elas \code{argparse} para tratar
argumentos de linha de comando, \code{pickle} para serializar estruturas de
dados e salvar os modelos,

\subsection{Uso}
Para rodar o programa, digite na linha de comandos

\begin{lstlisting}[language=bash]
$ python3 grammar.py input_file
\end{lstlisting}

em que \code{input_file} é o arquivo do corpus contendo a gramática que se
deseja extrair.

\section{Estratégias}

\section{Resultados}
As 1000 regras mais comuns representam 87.57\% das regras presentes no corpus e
podem ser vistas no arquivo \code{rules.txt}.

\end{document}
