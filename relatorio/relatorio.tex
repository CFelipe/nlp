\documentclass[12pt, a4paper]{article}
\usepackage[utf8]{inputenc}
\usepackage[brazilian]{babel} % Hifenização e dicionário
\usepackage[left=3.00cm, right=2.00cm, top=3.00cm, bottom=2.00cm]{geometry}
\usepackage{enumitem} % Para itemsep etc
\usepackage{longtable} % Dependência do longtabu
\usepackage{tabu} % Para melhor criação de tabelas
\usepackage{zi4} % Para fonte de códigos
\usepackage{listings} % Para códigos
\usepackage{lstautogobble} % Códigos indentados corretamente
\usepackage{color} % Para coloração de códigos
\usepackage{mathpazo} % Palatino
\usepackage{parskip} % Linha em branco entre parágrafos em vez de recuo
\usepackage{graphicx}
\usepackage{verbatim} % Para comentários
\usepackage{booktabs}
\usepackage{amsmath} % argmax
\usepackage[breaklinks]{hyperref}

\DeclareGraphicsExtensions{.pdf,.png}

\newcommand{\code}[1]{{\lstinline{#1}}}

\usepackage{listings}
\lstset{
    autogobble,
    columns=fullflexible,
    showspaces=false,
    showtabs=false,
    breaklines=true,
    showstringspaces=false,
    breakatwhitespace=true,
    escapeinside={(*@}{@*)},
    basicstyle=\ttfamily\footnotesize,
    frame=l,
    framesep=12pt,
    xleftmargin=12pt,
    tabsize=4,
    captionpos=b
}

\DeclareMathOperator*{\argmax}{arg\,max}
\DeclareMathOperator*{\argmin}{arg\,min}

\begin{document}
\begin{center}
    \textsc{Universidade Federal do Rio Grande do Norte} \\
    \textsc{Departamento de Informática e Matemática Aplicada}
\end{center}

\bigskip

\begin{tabular}{@{}rl@{}}
    \emph{Disciplina} & Processamento de Linguagem Natural \\
    \emph{Docente}    & Carlos Augusto Prolo \\
    \emph{Discente}   & Felipe Cortez de Sá \\
\end{tabular}

\bigskip

\begin{center}
\large Parts-of-speech tagger
\end{center}

\section{Introdução}
Este relatório descreve a implementação de dois algoritmos para realizar
\emph{parts-of-speech tagging} e os resultados obtidos utilizando o EVALB para
avaliação. 

\section{Programa}
O programa foi desenvolvido utilizando a linguagem de programação Python 3.
Antes de realizar o \emph{tagging}, é preciso treinar o modelo para cada
estratégia. O programa criará um diretório \code{models} onde os modelos são
salvos em arquivos \code{.pkl}, formato \emph{pickle} que serializa estruturas
de dados e podem ser posteriormente recuperados em outras execuções. Para rodar
o programa, digite na linha de comandos

\begin{lstlisting}[language=bash]
$ python3 [-a naive | viterbi] [-t "treebank" | -s "frase" | -i "arquivo com frases"]
\end{lstlisting}

A flag \code{-a} é utilizada para selecionar o algoritmo desejado e as opções
são \emph{naive} e \emph{viterbi}. Caso a flag seja omitida, o algoritmo
padrão é o Viterbi.

A flag \code{-t} treina o algoritmo selecionado (ingênuo ou Viterbi) com o
\emph{treebank} do corpus passado como argumento e salva o modelo em um arquivo \code{.pkl}.

A flag \code{-s} executa o tagger para uma frase passada como argumento.

A flag \code{-i} executa o tagger para o arquivo de entrada com uma frase por
linha. O programa escreverá os resultados do \emph{tagging} no arquivo de mesmo
nome da entrada com o sufixo \code{.tst}.

Os algoritmos implementados são derivados da classe \code{Tagger} definida em
\code{common.py} e devem implementar os métodos abstratos \code{train},
\code{load_model} e \code{tag_sentence}. A função \code{tag_file} é implementada
pela classe base e chama \code{tag_sentence} em todas as frases de
um arquivo de entrada.

\section{Estratégias}
\subsection{Algoritmo ingênuo}
\subsubsection{Treinamento}
Para armazenar as partes do discurso mais comuns para cada palavra, um
\emph{hash map} é indexado por palavras e partes do discurso, e o valor é a
quantidade de ocorrências de cada parte para cada palavra.
\code{words_pos["word"]["NN"]}, por exemplo, retorna a quantidade de terminais
na árvore de treino com formato \code{(NN word)}.

A função \code{tag_sentence} escolhe para cada palavra a \emph{tag}
com maior frequência, ou seja, $ \argmax_{tag} \text{words\_pos[word][tag]} $ .
Caso a palavra não seja encontrada, é escolhida a \emph{tag} mais comum ``NN''.

\subsection{Viterbi}
\subsubsection{Treinamento}
São calculados os bigramas, adicionados ao \emph{hash map} \code{bigrams}, e as
frequências de palavras para cada parte do discurso no \emph{hash map}
\code{pos_words}. Em vez de \code{words_pos[word][tag]}, a ordem é invertida e
utiliza-se \code{pos_words[tag][word]}. Ao terminar a contagem de frequências,
os dois \emph{hash maps} são normalizados, transformando as contagens em
probabilidades.

Para calcular os bigramas para começo e fim de frase, foram criadas as partes do
discurso \code{START} e \code{END}.

\subsubsection{Tagging}
O \emph{hash map} \code{viterbi} armazena os valores do algoritmo de Viterbi e o
\emph{hash map} \code{backpointer} armazena os ponteiros para determinar a
sequência mais provável no final do algoritmo.

Uma dificuldade encontrada foi a baixa ocorrência de algumas palavras no
corpus. Uma palavra rara resultava em baixas probabilidades, podendo zerar um
passo da matriz. Como os passos seguintes são calculados com base nos
anteriores, o erro era propagado, sendo impossível determinar \emph{tags} para
as palavras a partir daquele ponto.

A medida inicial tomada para tratar o problema foi repetir as probabilidades do
passo anterior caso fosse detectado que o passo não tinha probabilidades maiores
que zero.

Uma segunda estratégia foi identificar, no treinamento, quais palavras ocorriam
menos de $ x $ vezes no corpus, sendo $ x $ determinado pela constante
\code{UNKNOWN_TRESHOLD}. As palavras raras encontradas são contabilizadas como uma
única palavra com \emph{string} dada pela constante \code{UNKNOWN_STR}. No
processo de \emph{tagging}, caso uma palavra não seja encontrada no corpus, ela
é tratada como \code{UNKNOWN_STR}, recebendo a probabilidade de ter certa \emph{tag} dado que é uma palavra rara.

\section{Resultados}
Utilizou-se o corpus do Penn Treebank para treinamento, desenvolvimento e
validação. Os algoritmos foram treinados utilizando o arquivo \code{traindata},
ajustes no algoritmo foram feitos com o arquivo \code{bank0} e as árvores em \code{section23}
foram convertidas em frases com o programa \code{fromTreeToSentence}.
Utilizou-se então o EVALB para comparar o resultado do \emph{tagger} utilizando
as frases da \code{section23} para comparar com o \emph{gold standard} \code{section23}.

As saídas do EVALB estão no diretório \code{results}.

\subsection{Ingênuo}
90.88\% com EVALB.

\subsection{Viterbi}
93.64\% sem heurística, 94.33\% com heurística de palavras desconhecidas usando o Penn Treebank.

\end{document}
